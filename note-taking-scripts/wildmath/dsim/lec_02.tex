\lecture{2}{Wed 29 May 2024 21:18}{Sets and other data structures in mathematics}

Set theory is an attempt to \underline{organize mathematical objects}. This is parallel to attempts to organize \underline{physical} or \underline{real-life objects}. There are obvious connections to computer science.

\begin{definition}[Generic set]
    A set is unordered and without repetitions.
\end{definition}

\vspace{1em}  % Add some vertical space between definitions

There are four types of data structures:

\[
\begin{array}{|c|c|c|}
    \hline
    & \textbf{Unordered} & \textbf{Ordered} \\
    \hline
    \textbf{Without Repetition} & \text{set } \{1\, 2\, 3\} & \text{ordered set } \{1, 2, 3\} \\
    \hline
    \textbf{With Repetition} & \text{multiset } [1\, 1\, 4\, 2] & \text{list } [1, 1, 4, 2] \\
    \hline
\end{array}
\]

Notice the different notation where no commas suggest no order and braces suggest repetitions.

\[
    \fbox{
    \parbox{0.5\textwidth}{
        \begin{tabular}{@{}l@{}}
        \textbf{\{ \} -- Without repetitions} \\
        \textbf{[ ] -- With repetitions} \\
        \textbf{Space delimiter -- Unordered} \\
        \textbf{Comma delimiter -- Ordered}
        \end{tabular}
    }
}

\]

\[
\begin{aligned}
A &= \underset{\text{set}}{\{1\:2\:3\}} \\
H &= \underset{\text{oset}}{\{2, 7\}} \\
D &= \underset{\text{mset}}{[1\:5\:5\:4]} \\
F &= \underset{\text{list}}{[7, 5, 4]}
\end{aligned}
\]

\vspace{1em}  % Add some vertical space between definitions

\underline{Data structures in real life}

\begin{description}
    \item[\textbf{People in a company}:] set
    \item[\textbf{Names in a company}:] multiset (mset)
    \item[\textbf{Words in a dictionary}:] ordered set (oset)
    \item[\textbf{Letters in a word}:] list
    \item[\textbf{Webpages}:] set
    \item[\textbf{Images found on the internet}:] multiset (mset)
    \item[\textbf{Primes from 2 to 19}:] ordered set (oset)
    \item[\textbf{Bus numbers that arrive at a stop}:] list
\end{description}

While the kinds of elements in real life that appear in data structures are highly varied, we will start out with very focused examples.

\vspace{1em}

\[
\begin{aligned}
\textbf{Nat} &: \quad 1, 2, 3, 4, \ldots \\
\textbf{Nat}^{+} &: \quad 0, 1, 2, 3, \ldots
\end{aligned}
\]
For a precise theory, we are going to agree that the Hindu-Arabic form of a number
\( n \in Nat \) or \( k \in Nat^{+} \) is the only one allowed!

\eg{    \( n = 13 \textcolor{green}{\checkmark} \)

    \( n = 12+1 \textcolor{red}{\times}\)

    \( n = 1+1+1+1+1+1+4 \textcolor{red}{\times} \)

\( n = \text{smallest prime > 12} \textcolor{red}{\times} \)

    \( n = \text{Mary's lucky number}\textcolor{red}{\times} \)
}
\eg{
    \( k = 57109\textcolor{green}{\checkmark} \)\), \( k = 57.109\textcolor{red}{\times} \), \( k = 57.109 \cdot 0\textcolor{red}{\times} \)
}

Or first (cooked up) datastructure is not going to be a set.

\begin{definition}
    \underline{k-list}: For \( n,k \in \text{Nat} \) we define a k-list from n to be an
    expression of the form \( L = [m_{1},m_{2},\dots, m_{k}] \)
where each \( m_{i} \in \text{Nat} \) and \( 1 \le m_{i} \le n \).

\end{definition}

\vspace{1em}
This defines a new type object: \[
    \underline{List(n,k)}
.\]

\vspace{1em}
\eg{
    i) \( [3,2,4] \in \text{List}(4,3)\) \: ii) \( [13] \in \text{List}(15,1) \)

    iii) \( [1,5,5,1] \in \text{List}(5,4) \) \: iiii) \( [2,2,2,2] \in \text{List}(2,4) \)
}

\vspace{1em}
The objects (natural numbers) \( m_{1},\dots,m_{k} \) in the k-list L=\( [m_{1},m_{2},\dots,m_{k}] \) are called
the elements of L. They are considred as distinct elements, even if as numbers they are the same/equal.

\eg{ The 4-list  \( L=[3,1,3,2] \) has four elements, namely 3,1,3 and 2.
}
\vspace{1em}
The size of the k-list \( L = [m_{1},m_{2},\dots,m_{k}] \) is
\[
    \lvert L \rvert = k
.\]

The list the most fundamental data structure. An ordered set is a variant, where the elements
are required to be distinct.

\begin{definition}[k-ordered set]
    For \( n,k \in Nat \), a k-ordered set from n is an expression of the form
    \begin{displaymath}
        O = {m_{1},m_{2},\dots, m_{k}}
    \end{displaymath}

    where each \( m \in  \text{Nat} \), \( 1\le m_{i}\le n  \), and the \( m_{i} \) are
    all distinct, ie. \( m_{i} \neq m{j} \) if \( i \neq j \).

\end{definition}

\eg{
    \( \text{O}  = \{1,3,5\} \in \text{Oset}(5,3) \) with \( \abs{O} = \text{k} = 3 \).
}

Another variant of a list is a multiset, where order is no longer important.

\begin{definition}[k-multiset]
    For \( n,k \in \text{Nat} \), a k-multiset from n is an expression of the form
    \[
        M = [m_{1},m_{2}, \dots, m_{n}]
    \]
    where each \( m_{i} \in \text{Nat} \), \( 1 \le m_{i} \le n \), and the order
    of the elements \( m_{i} \) is unimportant.
\end{definition}

\eg{\( M = [1\:3\:3\:1\:15] = [15\:1\:3\:1\:3] \in \text{Mset}(15,5)\:\text{with}\:\abs{M}=5\)}.

\begin{definition}[k-set]
    For \( n,k \in \text{Nat}\) we define a k-set from n to be an expression of the
    form \[
        S = \{m_{1},m_{2},\dots, m_{k}\}
    \]
    where each \( m_{1} \in \text{Nat} \), \( 1 \le m \le n \), and the \( m_{i} \) are
    all distinct, and their order is unimportant.
\end{definition}

\eg{\( S = \{13\:7\:2\} = \{2\:7\:13\} \in  \{13\:3\} \) with \( \abs{S} = 3 \)}.
